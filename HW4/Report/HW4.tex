\documentclass[12pt]{article}
\usepackage[paper=letterpaper,margin=2cm]{geometry}
\usepackage{amsmath,amssymb,amsfonts}
\usepackage{enumitem}
\usepackage{titling}
\usepackage{multirow}
\usepackage{xcolor}
\usepackage{float}
\usepackage{graphicx}
\usepackage{xcolor}
\usepackage[colorlinks=true, linkcolor=red]{hyperref}
\usepackage{subcaption} % For subfigures
\usepackage{adjustbox}  % For centering the bottom image
\usepackage{listings}
\usepackage{xcolor} % For setting colors
\usepackage{booktabs} % For better tables
\usepackage{threeparttable} % For table notes
\usepackage{pifont}    % for symbols

\usepackage{listings}
\usepackage{xcolor}

\definecolor{codegreen}{rgb}{0.0, 0.514, 0.325}      
\definecolor{codegray}{rgb}{0.75, 0.75, 0.75}    
\definecolor{codeblue}{rgb}{0.122, 0.467, 0.706}  
\definecolor{extraLightGray}{rgb}{0.98, 0.98, 0.98}
\definecolor{codepink}{rgb}{0.894, 0.0, 0.443}

\lstdefinestyle{mystyle}{
    backgroundcolor=\color{extraLightGray},
    commentstyle=\color{codegreen},
    keywordstyle=\color{codeblue},
    numberstyle=\tiny\color{codegray},
    stringstyle=\color{codepink},
    basicstyle=\ttfamily\footnotesize,
    breakatwhitespace=false,
    breaklines=true,
    captionpos=b,
    keepspaces=true,
    numbers=left,
    numbersep=5pt,
    showspaces=false,
    showstringspaces=false,
    showtabs=false,
    tabsize=2
}
\lstset{style=mystyle}

\setlength{\droptitle}{-6em}

\begin{document}

\begin{center}
Aprendizagem 2023\\
Homework I --- Group 003\\
(ist1107028, ist1107137)\vskip 1cm
\end{center}

\large{\textbf{Part I}: Pen and paper}\normalsize

\vspace{20pt}
\textbf{Consider the bivariate observations}

\begin{equation*}\{
    x_1 = \begin{bmatrix}
        1 \\
        0
    \end{bmatrix},
    x_2 = \begin{bmatrix}
        0 \\
        2
    \end{bmatrix},
    x_3 = \begin{bmatrix}
        3 \\
        -1
    \end{bmatrix}\}
\end{equation*}

\vspace{10pt}
\textbf{and the multivariate Gaussian mixture given by}

\[
\mathbf{u}_1 = \begin{bmatrix} 2 \\ -1 \end{bmatrix}, \quad
\mathbf{u}_2 = \begin{bmatrix} 1 \\ 1 \end{bmatrix}, \quad
\Sigma_1 = \begin{bmatrix} 4 & 1 \\ 1 & 4 \end{bmatrix}, \quad
\Sigma_2 = \begin{bmatrix} 2 & 0 \\ 0 & 2 \end{bmatrix}, \quad
\pi_1 = 0.5, \quad \pi_2 = 0.5
\]

\vspace{10pt}
\textbf{Answer the following questions by presenting all intermediary steps, and use 3 decimal places in
each.}

\begin{enumerate}
    \item \textbf{Perform two epochs of the EM clustering algorithm and determine the new parameters.}
    
    \vspace{10pt}
    \colorbox{codeblue}{\textcolor{white}{FIRST EM EPOCH}}

    \begin{enumerate}[label=\arabic*)]
        \item Expectation: \textbf{\textcolor{codeblue}{E-STEP}}
        
        \vspace{10pt}
        \fbox{$x_1$}

        \vspace{10pt}
        \begin{itemize}[label=\ding{222}]
            \item Cluster $c=1$:
            \begin{equation*}
                \begin{aligned}
                    &\text{prior: } P(c=1) = \pi_1 = \mathbf{0.5} \\
                    \\
                    &\text{likelihood: } p(x_1|c=1) = \mathcal{N}(x_1| \mu_1, \sigma_1) \\
                    &= \frac{1}{(2 \cdot \pi)} \cdot \frac{1}{det(\Sigma_1)} \cdot \exp \left( -\frac{1}{2} \cdot (x_1 - \mu_1)^{T} \Sigma_1^{-1} \cdot (x_1 - \mu_1) \right)\\
                    &= \frac{1}{(2 \cdot \pi)} \cdot \frac{1}{15} \cdot \exp \left( -\frac{1}{2} \cdot \left(\begin{pmatrix}
                    1\\
                    0
                    \end{pmatrix} - \begin{pmatrix}
                    2\\
                    -1
                    \end{pmatrix}\right)^{T} \cdot \begin{pmatrix}
                    4 & 1\\
                    1 & 4
                    \end{pmatrix}^{-1} \cdot \left(\begin{pmatrix}
                    1\\
                    0
                    \end{pmatrix} - \begin{pmatrix}
                    2\\
                    -1
                    \end{pmatrix}\right) \right)\\
                    &= \frac{1}{(2 \cdot \pi)} \cdot \frac{1}{15} \cdot \exp \left( -\frac{1}{2} \cdot \begin{pmatrix}
                    -1 & 1\\
                    \end{pmatrix} \cdot \begin{pmatrix}
                    4 & 1\\
                    1 & 4
                    \end{pmatrix} \cdot \begin{pmatrix}
                    -1\\
                    1
                    \end{pmatrix} \right) = \frac{e^{-3}}{2\pi \cdot 15} = \mathbf{0.029}\\
                    \\
                    &\text{joint probability: } P(c=1, x_1) =  P(c=1)p(x_1|c=1) = \pi_1 \cdot \mathcal{N}(x_1| \mu_1, \sigma_1) = \mathbf{0.015}\\
                    \\
                    &\text{normalized posterior: } P(c=1|x_1) = \frac{0.015}{0.015+0.007} = \mathbf{0.681}
                \end{aligned}
            \end{equation*}

            \item Cluster $c=2$:
            \begin{equation*}
                \begin{aligned}
                    &\text{prior: } P(c=2) = \pi_2 = \mathbf{0.5} \\
                    \\
                    &\text{likelihood: } p(x_1|c=2) = \mathcal{N}(x_1| \mu_2, \sigma_2) \\
                    &= \frac{1}{(2 \cdot \pi)} \cdot \frac{1}{det(\Sigma_2)} \cdot \exp \left( -\frac{1}{2} \cdot (x_1 - \mu_2)^{T} \Sigma_2^{-1} \cdot (x_1 - \mu_2) \right)\\
                    &= \frac{1}{(2 \cdot \pi)} \cdot \frac{1}{4} \cdot \exp \left( -\frac{1}{2} \cdot \left(\begin{pmatrix}
                    1\\
                    0
                    \end{pmatrix} - \begin{pmatrix}
                    1\\
                    1
                    \end{pmatrix}\right)^{T} \cdot \begin{pmatrix}
                    2 & 0\\
                    0 & 2
                    \end{pmatrix}^{-1} \cdot \left(\begin{pmatrix}
                    1\\
                    0
                    \end{pmatrix} - \begin{pmatrix}
                    1\\
                    1
                    \end{pmatrix}\right) \right)\\
                    &= \frac{1}{(2 \cdot \pi)} \cdot \frac{1}{4} \cdot \exp \left( -\frac{1}{2} \cdot \begin{pmatrix}
                    0 & -1\\
                    \end{pmatrix} \cdot \begin{pmatrix}
                    2 & 0\\
                    0 & 2
                    \end{pmatrix} \cdot \begin{pmatrix}
                    0\\
                    -1
                    \end{pmatrix} \right) = \frac{e^{-1}}{2 \pi \cdot 4} = \mathbf{0.015}\\
                    \\
                    &\text{joint probability: } P(c=2, x_1) =  P(c=2)p(x_1|c=2) = \pi_2 \cdot \mathcal{N}(x_1| \mu_2, \sigma_2) = \mathbf{0.007}\\
                    \\
                    &\text{normalized posterior: } P(c=2|x_1) = \frac{0.007}{0.007+0.015} = \mathbf{0.318}
                \end{aligned}
            \end{equation*}
        \end{itemize}
        
        \vspace{10pt}
        \fbox{$x_2$}

        \vspace{10pt}
        \begin{itemize}[label=\ding{222}]
            \item Cluster $c=1$:
                
            \begin{equation*}
                \begin{aligned}
                    &\text{prior: } P(c=1) = \pi_1 = \mathbf{0.5} \\
                    &\text{likelihood: } p(x_2|c=1) = \mathcal{N}(x_2| \mu_1, \sigma_1) \\
                    &= \frac{1}{(2 \cdot \pi)} \cdot \frac{1}{det(\Sigma_1)} \cdot \exp \left( -\frac{1}{2} \cdot (x_2 - \mu_1)^{T} \Sigma_1^{-1} \cdot (x_2 - \mu_1) \right)\\
                    &= \frac{1}{(2 \cdot \pi)} \cdot \frac{1}{15} \cdot \exp \left( -\frac{1}{2} \cdot \left(\begin{pmatrix}
                    0\\
                    2
                    \end{pmatrix} - \begin{pmatrix}
                    2\\
                    -1
                    \end{pmatrix}\right)^{T} \cdot \begin{pmatrix}
                    4 & 1\\
                    1 & 4
                    \end{pmatrix}^{-1} \cdot \left(\begin{pmatrix}
                    0\\
                    2
                    \end{pmatrix} - \begin{pmatrix}
                    2\\
                    -1
                    \end{pmatrix}\right) \right)\\
                    &= \frac{1}{(2 \cdot \pi)} \cdot \frac{1}{15} \cdot \exp \left( -\frac{1}{2} \cdot \begin{pmatrix}
                    -2 & 3\\
                    \end{pmatrix} \cdot \begin{pmatrix}
                    4 & 1\\
                    1 & 4
                    \end{pmatrix} \cdot \begin{pmatrix}
                    -2\\
                    3
                    \end{pmatrix} \right) = \frac{e^{-20}}{2 \pi \cdot 15} = \mathbf{0}\\ 
                    \\    
                    &\text{joint probability: } P(c=1, x_2) =  P(c=1)p(x_2|c=1) = \pi_1 \cdot \mathcal{N}(x_2| \mu_1, \sigma_1) = \mathbf{0}\\
                    \\
                    &\text{normalized posterior: }P(c=1|x_2) = \frac{0}{0+0.003} = \mathbf{0}
                \end{aligned}
            \end{equation*}

            \newpage
            \item Cluster $c=2$:
            
            \begin{equation*}
                \begin{aligned}
                    &\text{prior: } P(c=2) = \pi_2 = \mathbf{0.5} \\
                    &\text{likelihood: } p(x_2|c=2) = \mathcal{N}(x_2| \mu_2, \sigma_2) \\
                    &= \frac{1}{(2 \cdot \pi)} \cdot \frac{1}{det(\Sigma_2)} \cdot \exp \left( -\frac{1}{2} \cdot (x_2 - \mu_2)^{T} \Sigma_2^{-1} \cdot (x_2 - \mu_2) \right)\\
                    &= \frac{1}{(2 \cdot \pi)} \cdot \frac{1}{4} \cdot \exp \left( -\frac{1}{2} \cdot \left(\begin{pmatrix}
                    0\\
                    2
                    \end{pmatrix} - \begin{pmatrix}
                    1\\
                    1
                    \end{pmatrix}\right)^{T} \cdot \begin{pmatrix}
                    2 & 0\\
                    0 & 2
                    \end{pmatrix}^{-1} \cdot \left(\begin{pmatrix}
                    0\\
                    2
                    \end{pmatrix} - \begin{pmatrix}
                    1\\
                    1
                    \end{pmatrix}\right) \right)\\
                    &= \frac{1}{(2 \cdot \pi)} \cdot \frac{1}{4} \cdot \exp \left( -\frac{1}{2} \cdot \begin{pmatrix}
                    -1 & 1\\
                    \end{pmatrix} \cdot \begin{pmatrix}
                    2 & 0\\
                    0 & 2
                    \end{pmatrix} \cdot \begin{pmatrix}
                    -1\\
                    1
                    \end{pmatrix} \right) = \frac{e^{-2}}{2 \pi \cdot 4} = \mathbf{0.005}\\
                    \\
                    &\text{joint probability: } P(c=2, x_2) =  P(c=2)p(x_2|c=2) = \pi_2 \cdot \mathcal{N}(x_2| \mu_2, \sigma_2) = \mathbf{0.003}\\
                    \\
                    &\text{normalized posterior: } P(c=2|x_2) = \frac{0.003}{0 + 0.003} = \mathbf{1}
                \end{aligned}
            \end{equation*}
        \end{itemize}

        \vspace{10pt}
        \fbox{$x_3$}

        \vspace{10pt}
        \begin{itemize}[label=\ding{222}]
            \item Cluster $c=1$:
                
            \begin{equation*}
                \begin{aligned}
                    &\text{prior: } P(c=1) = \pi_1 = \mathbf{0.5} \\
                    &\text{likelihood: } p(x_3|c=1) = \mathcal{N}(x_3| \mu_1, \sigma_1) \\
                    &= \frac{1}{(2 \cdot \pi)} \cdot \frac{1}{det(\Sigma_1)} \cdot \exp \left( -\frac{1}{2} \cdot (x_3 - \mu_1)^{T} \Sigma_1^{-1} \cdot (x_3 - \mu_1) \right)\\
                    &= \frac{1}{(2 \cdot \pi)} \cdot \frac{1}{15} \cdot \exp \left( -\frac{1}{2} \cdot \left(\begin{pmatrix}
                    3\\
                    -1
                    \end{pmatrix} - \begin{pmatrix}
                    2\\
                    -1
                    \end{pmatrix}\right)^{T} \cdot \begin{pmatrix}
                    4 & 1\\
                    1 & 4
                    \end{pmatrix}^{-1} \cdot \left(\begin{pmatrix}
                    3\\
                    -1
                    \end{pmatrix} - \begin{pmatrix}
                    2\\
                    -1
                    \end{pmatrix}\right) \right)\\
                    &= \frac{1}{(2 \cdot \pi)} \cdot \frac{1}{15} \cdot \exp \left( -\frac{1}{2} \cdot \begin{pmatrix}
                    1 & 0\\
                    \end{pmatrix} \cdot \begin{pmatrix}
                    4 & 1\\
                    1 & 4
                    \end{pmatrix} \cdot \begin{pmatrix}
                    1\\
                    0
                    \end{pmatrix} \right) = \frac{e^{-2}}{2 \pi \cdot 15} = \mathbf{0.001}\\
                    \\
                    &\text{joint probability: } P(c=1, x_3) =  P(c=1)p(x_3|c=1) = \pi_1 \cdot \mathcal{N}(x_3| \mu_1, \sigma_1) = \mathbf{0.001}\\
                    \\
                    &\text{normalized posterior: } P(c=1|x_3)= \frac{0.001}{0+0.001} = \mathbf{1}
                \end{aligned}
            \end{equation*}

            \newpage
            \item Cluster $c=2$:
            
            \begin{equation*}
                \begin{aligned}
                    &\text{prior: } P(c=2) = \pi_2 = \mathbf{0.5} \\
                    &\text{likelihood: } p(x_3|c=2) = \mathcal{N}(x_3| \mu_2, \sigma_2) \\
                    &= \frac{1}{(2 \cdot \pi)} \cdot \frac{1}{det(\Sigma_2)} \cdot \exp \left( -\frac{1}{2} \cdot (x_3 - \mu_2)^{T} \Sigma_2^{-1} \cdot (x_3 - \mu_2) \right)\\
                    &= \frac{1}{(2 \cdot \pi)} \cdot \frac{1}{4} \cdot \exp \left( -\frac{1}{2} \cdot \left(\begin{pmatrix}
                    3\\
                    -1
                    \end{pmatrix} - \begin{pmatrix}
                    1\\
                    1
                    \end{pmatrix}\right)^{T} \cdot \begin{pmatrix}
                    2 & 0\\
                    0 & 2
                    \end{pmatrix}^{-1} \cdot \left(\begin{pmatrix}
                    3\\
                    -1
                    \end{pmatrix} - \begin{pmatrix}
                    1\\
                    1
                    \end{pmatrix}\right) \right)\\
                    &= \frac{1}{(2 \cdot \pi)} \cdot \frac{1}{4} \cdot \exp \left( -\frac{1}{2} \cdot \begin{pmatrix}
                    2 & -2\\
                    \end{pmatrix} \cdot \begin{pmatrix}
                    2 & 0\\
                    0 & 2
                    \end{pmatrix} \cdot \begin{pmatrix}
                    2\\
                    -2
                    \end{pmatrix} \right) = \frac{e^{-8}}{2 \pi \cdot 4} = \mathbf{0}\\
                    \\
                    &\text{joint probability: } P(c=1, x_3) =  P(c=1)p(x_3|c=1) = \pi_1 \cdot \mathcal{N}(x_3| \mu_1, \sigma_1) = \mathbf{0}\\
                    \\
                    &\text{normalized posterior: } P(c=2|x_3) = \frac{0}{0+0.001} = \mathbf{0}
                \end{aligned}
            \end{equation*}
        \end{itemize}


        \begin{table}[H]
            \begin{center}
                \begin{threeparttable}
                \begin{tabular}{c|c|c}
                    Observations & $c=1$ & $c=2$ \\
                    \hline
                    $x_1$ & $0.681$ & $0.318$\\
                    $x_2$ & $0$ & $1$\\
                    $x_3$ & $1$ & $0$\\
                \end{tabular}
                \caption{Normalized posteriors}
                \end{threeparttable}
            \end{center}
        \end{table}

        \vspace{10pt}
        \item Maximization: \textbf{\textcolor{codeblue}{M-STEP}}
        
        \vspace{10pt}
        For the recalculation we will use the following formulas (n represents the cluster):

        \vspace{10pt}
        For the means:
        \begin{equation}\label{means}
            \mu_\text{n} = \frac{P(c=n|x_1) \cdot x_1 + P(c=n|x_2) \cdot x_2 + P(c=n|x_3) \cdot x_3}{P(c=n|x_1) + P(c=n|x_2) + P(c=n|x_3)}
        \end{equation}

        \vspace{10pt}
        For the covariance matrices:

        \begin{equation}\label{covariance_matrix}
            \Sigma_n = \begin{pmatrix}
                \Sigma_{11} & \Sigma_{21} \\
                \Sigma_{12} & \Sigma_{22} \\
            \end{pmatrix}
        \end{equation}
        
        Where
        \begin{equation*}
            \resizebox{0.88\hsize}{!}{$
            \Sigma_{ij} = \frac{
              P(c=n|x_1)\left(\left(x_{1i} - \mu_{1i}\right) \left(x_{1j}- \mu_{1j}\right)\right) 
            + P(c=n|x_2)\left(\left(x_{2i} - \mu_{2i}\right) \left(x_{2j}- \mu_{2j}\right)\right) 
            + P(c=n|x_3)\left(\left(x_{3i} - \mu_{3i}\right) \left(x_{3j}- \mu_{3j}\right)\right)
            }{P(c=n|x_1) + P(c=n|x_2) + P(c=n|x_3)}
            $}
        \end{equation*}
        
        \newpage
        \begin{itemize}[label=\ding{222}]
            \item Cluster $c=1$:
        \begin{equation*}
            \footnotesize
            \mu_1 = \frac{0.681 \cdot x_1 + 0 \cdot x_2 + 1 \cdot x_3}{0.681+0+1} = \frac{0.681 \cdot \begin{pmatrix}
            1\\
            0
            \end{pmatrix}+ 0 \cdot \begin{pmatrix}
            0\\
            2
            \end{pmatrix} + 1 \cdot \begin{pmatrix}
            3\\
            -1
            \end{pmatrix}}{1.681} = \begin{pmatrix}
            2.190\\
            -0.595
            \end{pmatrix}\\
        \end{equation*}

        \begin{equation*}
            \Sigma_1 = \begin{pmatrix}
                \Sigma_{11} & \Sigma_{21} \\
                \Sigma_{12} & \Sigma_{22}
            \end{pmatrix} = \begin{pmatrix}
                0.964 & -0.482 \\
                -0.482 & 0.241
            \end{pmatrix}
        \end{equation*}

        \begin{equation*}
            \fontsize{9.5pt}{10pt}\selectfont
            \begin{aligned}
                \Sigma_{11} &= \frac{0.681 \cdot \left((x_{11}-\mu_{11}) \cdot (x_{11} - \mu_{11})\right)+ 0 \cdot \left((x_{21}-\mu_{11}) \cdot (x_{21} - \mu_{11})\right)+ 1 \cdot \left((x_{31}-\mu_{11}) \cdot (x_{31} - \mu_{11})\right)}{1.681}\\
                &=\frac{0.681 \cdot \left((1-2.190) \cdot (1 - 2.190)\right) + 1 \cdot \left((3-2.190) \cdot (3 - 2.190)\right)}{1.681} = \frac{0.964 + 0.656}{1.681} = 0.964\\
                \\
                \Sigma_{21} &= \frac{0.681 \cdot \left((x_{12}-\mu_{12}) \cdot (x_{11} - \mu_{11})\right)+ 0 \cdot \left((x_{22}-\mu_{12}) \cdot (x_{21} - \mu_{11})\right)+ 1 \cdot \left((x_{32}-\mu_{12}) \cdot (x_{31} - \mu_{11})\right)}{1.681}\\
                &=\frac{0.681 \cdot \left((0-(-0.595)) \cdot (1 - 2.190)\right) + 1 \cdot \left((-1-(-0.595)) \cdot (3 - 2.190)\right)}{1.681} = \frac{-0.482 + -0.328}{1.681} \\
                &= -0.482\\
                \\
                \Sigma_{12} &= \Sigma_{21} = -0.482\\
                \\
                \Sigma_{22} &= \frac{0.681 \cdot \left((x_{12}-\mu_{12}) \cdot (x_{12} - \mu_{12})\right)+ 0 \cdot \left((x_{22}-\mu_{12}) \cdot (x_{22} - \mu_{12})\right)+ 1 \cdot \left((x_{32}-\mu_{12}) \cdot (x_{32} - \mu_{12})\right)}{1.681}\\
                &=\frac{0.681 \cdot \left((0-(-0.595)) \cdot (0-(-0.595))\right) + 1 \cdot \left((-1-(-0.595)) \cdot (-1-(-0.595))\right)}{1.681} = \frac{0.241 + 0.164}{1.681}\\
                &= 0.241\\
            \end{aligned}
        \end{equation*}

    \vspace{20pt}
    \item{cluster $c=2$}
    \begin{equation*}
        \footnotesize
        \mu_2 = \frac{0.318 \cdot x_1 + 1 \cdot x_2 + 0 \cdot x_3}{0.318+1+0} = \frac{0.318 \cdot \begin{pmatrix}
        1\\
        0
        \end{pmatrix}+ 0 \cdot \begin{pmatrix}
        0\\
        2
        \end{pmatrix} + 1 \cdot \begin{pmatrix}
        3\\
        -1
        \end{pmatrix}}{1.318} = \begin{pmatrix}
        2.517\\
        -0.759
        \end{pmatrix}\\
    \end{equation*}

    \begin{equation*}
        \Sigma_2 = \begin{pmatrix}
            \Sigma_{11} & \Sigma_{21} \\
            \Sigma_{12} & \Sigma_{22}
        \end{pmatrix} = \begin{pmatrix}
            5.362 & -5.546 \\
            -5.546 & 7.795
        \end{pmatrix}
    \end{equation*}

    \begin{equation*}
        \fontsize{9.5pt}{10pt}\selectfont
        \begin{aligned}
            \Sigma_{11} &= \frac{0.318 \cdot \left((x_{11}-\mu_{21}) \cdot (x_{11} - \mu_{21})\right)+ 1 \cdot \left((x_{21}-\mu_{21}) \cdot (x_{21} - \mu_{21})\right)+ 0 \cdot \left((x_{31}-\mu_{21}) \cdot (x_{31} - \mu_{21})\right)}{1.318}\\
            &=\frac{0.318 \cdot \left((1-2.517) \cdot (1-2.517)\right) + 1 \cdot \left((0-2.517) \cdot (0 - 2.517)\right)}{1.318} = \frac{0.732+6.335}{1.318} = 5.362\\
            \\
            \Sigma_{21} &= \frac{0.318 \cdot \left((x_{12}-\mu_{22}) \cdot (x_{11} - \mu_{21})\right)+ 1 \cdot \left((x_{22}-\mu_{22}) \cdot (x_{21} - \mu_{21})\right)+ 0 \cdot \left((x_{32}-\mu_{22}) \cdot (x_{31} - \mu_{21})\right)}{1.318}\\
            &=\frac{0.318 \cdot \left((0-(-0.759)) \cdot (1-2.517)\right) + 1 \cdot \left((2-(-0.759)) \cdot (0 - 2.517)\right)}{1.318} = \frac{-0.366+(-6.944)}{1.318} \\
            &= -5.546\\
            \\
            &\Sigma_{12} = \Sigma_{21} = -5.546\\
        \end{aligned}
    \end{equation*}

    \begin{equation*}
        \fontsize{9.5pt}{10pt}\selectfont
        \begin{aligned}
        \Sigma_{22} &= \frac{0.318 \cdot \left((x_{12}-\mu_{22}) \cdot (x_{12} - \mu_{22})\right)+ 1 \cdot \left((x_{22}-\mu_{22}) \cdot (x_{22} - \mu_{22})\right)+ 0 \cdot \left((x_{32}-\mu_{22}) \cdot (x_{32} - \mu_{22})\right)}{1.318}\\
        &= \frac{0.318 \cdot \left((0-(-0.759)) \cdot (0-(-0.759))\right) + 1 \cdot \left((2-(-0.759)) \cdot (2 - (-0.759))\right)}{1.318} \\
        &= \frac{0.183+7.612}{1.318} = 7.795\\
        \end{aligned}
    \end{equation*}

    \end{itemize}

    Normalized priors:

    \begin{equation*}
        \begin{aligned}
            P(c=1) &= \frac{0.681+0+1}{(0.681+0+1) + (0.318+1+0)} = 0.561\\
            P(c=2) &= \frac{0.318+1+0}{(0.681+0+1) + (0.318+1+0)} = 0.439\\
        \end{aligned}
    \end{equation*}

    \end{enumerate}

    \vspace{10pt}
    \item \textbf{Using the final parameters computed in previous question:}
    \begin{enumerate}[label=\alph*)]
        \item \textbf{perform a hard assignment of observations to clusters under a MAP assumption.}
        \item \textbf{compute the silhouette of the larger cluster (the one that has more observations
        assigned to it) using the Euclidean distance.}
    \end{enumerate}
\end{enumerate}

\vspace{20pt}
\large{\textbf{Part II}: Programming}\normalsize

\vspace{20pt}
\textbf{In the next exercise you will use the \texttt{accounts.csv} dataset. This dataset contains account details
of bank clients, and the target variable y is binary ('has the client subscribed a term deposit?').}

\begin{enumerate}
    \item \textbf{Select the first 8 features and remove duplicates and null values. Normalize the data
    using \texttt{MinMaxScaler}. Using \texttt{sklearn}, apply \texttt{k-means clustering} (without targets) on the
    normalized data with \( k = \{2,3,4,5,6,7,8\} \). Apply k-means randomly initialized, using \texttt{max\_iter = 500}
    and \texttt{random\_state = 42}. Plot the different sum of squared errors (SSE) using the
    \texttt{\_inertia} attribute of \texttt{k-means} according to the number of clusters.}
    
    \textbf{Hint:} You can use \texttt{get\_dummies()} to change the feature type from categorical to numerical
    (e.g. \texttt{pd.get\_dummies(data, drop\_first=True)})

    \item \textbf{According to the previous plot, how many underlying customer segments (clusters)
    should there be ? Explain based on the trade off between the clusters and inertia.}

    \item \textbf{Would k-modes be a better clustering approach ? Explain why based on the dataset
    features.}

    \item \textbf{Apply PCA to the data :}
    
    \begin{enumerate}[label=\alph*)]
        \item \textbf{Use \texttt{StandardScaler} to scale the data before you apply \texttt{fit\_transform}. How much
        variability is explained by the top 2 components ?}
        \item \textbf{Provide a scatterplot according to the first 2 principal components and color the
        points according to $k=3$ clusters. Can we clearly separate the clusters ? Justify.}
    \end{enumerate}
    
    \item \textbf{Plot the cluster conditional features of the frequencies of `job" and `education"
    according to k-means, with \texttt{multiple='dodge'}, \texttt{stat='density'}, \texttt{shrink=0.8},
    \texttt{common\_norm=False}. Analyze the frequency plots using \texttt{sns.displot}, (see Data Exploration
    notebook). Describe the main differences between the clusters in no more than half page.}








\end{enumerate}
\end{document}
