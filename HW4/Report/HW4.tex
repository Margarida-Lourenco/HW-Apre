\documentclass[12pt]{article}
\usepackage[paper=letterpaper,margin=2cm]{geometry}
\usepackage{amsmath,amssymb,amsfonts}
\usepackage{enumitem}
\usepackage{titling}
\usepackage{multirow}
\usepackage{xcolor}
\usepackage{float}
\usepackage{graphicx}
\usepackage{xcolor}
\usepackage[colorlinks=true, linkcolor=red]{hyperref}
\usepackage{subcaption} % For subfigures
\usepackage{adjustbox}  % For centering the bottom image
\usepackage{listings}
\usepackage{xcolor} % For setting colors
\usepackage{booktabs} % For better tables
\usepackage{threeparttable} % For table notes

\usepackage{listings}
\usepackage{xcolor}

\definecolor{codegreen}{rgb}{0.0, 0.514, 0.325}      
\definecolor{codegray}{rgb}{0.75, 0.75, 0.75}    
\definecolor{codeblue}{rgb}{0.122, 0.467, 0.706}  
\definecolor{extraLightGray}{rgb}{0.98, 0.98, 0.98}
\definecolor{codepink}{rgb}{0.894, 0.0, 0.443}

\lstdefinestyle{mystyle}{
    backgroundcolor=\color{extraLightGray},
    commentstyle=\color{codegreen},
    keywordstyle=\color{codeblue},
    numberstyle=\tiny\color{codegray},
    stringstyle=\color{codepink},
    basicstyle=\ttfamily\footnotesize,
    breakatwhitespace=false,
    breaklines=true,
    captionpos=b,
    keepspaces=true,
    numbers=left,
    numbersep=5pt,
    showspaces=false,
    showstringspaces=false,
    showtabs=false,
    tabsize=2
}
\lstset{style=mystyle}

\setlength{\droptitle}{-6em}

\begin{document}

\begin{center}
Aprendizagem 2023\\
Homework I --- Group 003\\
(ist1107028, ist1107137)\vskip 1cm
\end{center}

\large{\textbf{Part I}: Pen and paper}\normalsize

\vspace{20pt}
\textbf{Consider the bivariate observations}

\begin{equation*}\{
    x_1 = \begin{bmatrix}
        1 \\
        0
    \end{bmatrix},
    x_2 = \begin{bmatrix}
        0 \\
        2
    \end{bmatrix},
    x_3 = \begin{bmatrix}
        3 \\
        -1
    \end{bmatrix}\}
\end{equation*}

\vspace{10pt}
\textbf{and the multivariate Gaussian mixture given by}

\[
\mathbf{u}_1 = \begin{bmatrix} 2 \\ -1 \end{bmatrix}, \quad
\mathbf{u}_2 = \begin{bmatrix} 1 \\ 1 \end{bmatrix}, \quad
\Sigma_1 = \begin{bmatrix} 2 & 1 \\ 1 & 2 \end{bmatrix}, \quad
\Sigma_2 = \begin{bmatrix} 2 & 0 \\ 0 & 2 \end{bmatrix}, \quad
\pi_1 = 0.5, \quad \pi_2 = 0.5
\]

\vspace{10pt}
\textbf{Answer the following questions by presenting all intermediary steps, and use 3 decimal places in
each.}

\begin{enumerate}
    \item \textbf{Perform two epochs of the EM clustering algorithm and determine the new parameters.}
    \item \textbf{Using the final parameters computed in previous question:}
    \begin{enumerate}[label=\alph*)]
        \item \textbf{perform a hard assignment of observations to clusters under a MAP assumption.}
        \item \textbf{compute the silhouette of the larger cluster (the one that has more observations
        assigned to it) using the Euclidean distance.}
    \end{enumerate}
\end{enumerate}

\vspace{20pt}
\large{\textbf{Part II}: Programming}\normalsize

\vspace{20pt}
\textbf{In the next exercise you will use the \texttt{accounts.csv} dataset. This dataset contains account details
of bank clients, and the target variable y is binary ('has the client subscribed a term deposit?').}

\begin{enumerate}
    \item \textbf{Select the first 8 features and remove duplicates and null values. Normalize the data
    using \texttt{MinMaxScaler}. Using \texttt{sklearn}, apply \texttt{k-means clustering} (without targets) on the
    normalized data with \( k = \{2,3,4,5,6,7,8\} \). Apply k-means randomly initialized, using \texttt{max\_iter = 500}
    and \texttt{random\_state = 42}. Plot the different sum of squared errors (SSE) using the
    \texttt{\_inertia} attribute of \texttt{k-means} according to the number of clusters.}
    
    \textbf{Hint:} You can use \texttt{get\_dummies()} to change the feature type from categorical to numerical
    (e.g. \texttt{pd.get\_dummies(data, drop\_first=True)})

    \item \textbf{According to the previous plot, how many underlying customer segments (clusters)
    should there be ? Explain based on the trade off between the clusters and inertia.}

    \item \textbf{Would k-modes be a better clustering approach ? Explain why based on the dataset
    features.}

    \item \textbf{Apply PCA to the data :}
    
    \begin{enumerate}[label=\alph*)]
        \item \textbf{Use \texttt{StandardScaler} to scale the data before you apply \texttt{fit\_transform}. How much
        variability is explained by the top 2 components ?}
        \item \textbf{Provide a scatterplot according to the first 2 principal components and color the
        points according to $k=3$ clusters. Can we clearly separate the clusters ? Justify.}
    \end{enumerate}
    
    \item \textbf{Plot the cluster conditional features of the frequencies of `job" and `education"
    according to k-means, with \texttt{multiple='dodge'}, \texttt{stat='density'}, \texttt{shrink=0.8},
    \texttt{common\_norm=False}. Analyze the frequency plots using \texttt{sns.displot}, (see Data Exploration
    notebook). Describe the main differences between the clusters in no more than half page.}








\end{enumerate}
\end{document}
