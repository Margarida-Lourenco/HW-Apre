\documentclass[12pt]{article}
\usepackage[paper=letterpaper,margin=2cm]{geometry}
\usepackage{amsmath,amssymb,amsfonts}
\usepackage{enumitem}
\usepackage{titling}
\usepackage{multirow}
\usepackage{xcolor}
\usepackage{float}
\usepackage{graphicx}
\usepackage{xcolor}
\definecolor{ISTBlue}{RGB}{0, 139, 255}
\usepackage[colorlinks=true, linkcolor=red]{hyperref}
\usepackage{subcaption} % For subfigures
\usepackage{adjustbox}  % For centering the bottom image
\usepackage{listings}
\usepackage{xcolor} % For setting colors
\usepackage{booktabs} % For better tables
\usepackage{threeparttable} % For table notes

\usepackage{listings}
\usepackage{xcolor}

\definecolor{codegreen}{rgb}{0.0, 0.514, 0.325}      
\definecolor{codegray}{rgb}{0.75, 0.75, 0.75}    
\definecolor{codeblue}{rgb}{0.122, 0.467, 0.706}  
\definecolor{extraLightGray}{rgb}{0.98, 0.98, 0.98}
\definecolor{codepink}{rgb}{0.894, 0.0, 0.443}

\lstdefinestyle{mystyle}{
    backgroundcolor=\color{extraLightGray},
    commentstyle=\color{codegreen},
    keywordstyle=\color{codeblue},
    numberstyle=\tiny\color{codegray},
    stringstyle=\color{codepink},
    basicstyle=\ttfamily\footnotesize,
    breakatwhitespace=false,
    breaklines=true,
    captionpos=b,
    keepspaces=true,
    numbers=left,
    numbersep=5pt,
    showspaces=false,
    showstringspaces=false,
    showtabs=false,
    tabsize=2
}
\lstset{style=mystyle}

\setlength{\droptitle}{-6em}

\begin{document}

\begin{center}
Aprendizagem 2023\\
Homework I --- Group 003\\
(ist1107028, ist1107137)\vskip 1cm
\end{center}

\large{\textbf{Part I}: Pen and paper}\normalsize

\vspace{10pt}
\textbf{For questions in this group, show your numerical results with 5 decimals or scientific notation.
Hint: we highly recommend the use of \texttt{numpy} (e.g., \texttt{linalg.pinv} for inverse) or other programmatic
facilities to support the calculus involved in both questions (1) and (2).}

\textbf{Below is a training dataset $\mathbf{D}$ composed by two input variables and two output variables, one of 
which is numerical ($\mathbf{y_{num}}$) and the other categorical ($\mathbf{y_{class}}$) . Consider a polynomial basis function
$\mathbf{\phi(y_1,y_2) = y_1 \times y_2}$ that transforms the original space into a new one-dimensional space.}

\begin{table}[H]
    \begin{center}
        \begin{tabular}{c|cc|cc}
            D & $y_1$ & $y_2$ & $y_{\text{num}}$ & $y_{\text{class}}$ \\
            \hline
            $x_1$ & 1 & 1 & 1.25 & B \\
            $x_2$ & 1 & 3 & 7.0 & A \\
            $x_3$ & 3 & 2 & 2.7 & C \\
            $x_4$ & 3 & 3 & 3.2 & A \\
            $x_5$ & 2 & 4 & 5.5 & B \\
        \end{tabular}
    \end{center}
\end{table}

\begin{enumerate}
    \item \textbf{Learn a regression model on the transformed feature space using the OLS closed form
    solution to predict the continuous output variable $\mathbf{y_{\text{num}}}$.}
    
    \item \textbf{Repeat the previous exercise, but this time learn a Ridge regression with penalty
    factor $\mathbf{\lambda = 1}$. Compare the learnt coefficients with the ones from the previous exercise and
    discuss how regularization affects them.}

    \item \textbf{Given three new test observations and their corresponding output\\}
    $\mathbf{x_6 = (2,2,0.7), x_7 = (1,2,1.1),}$ \textbf{ and } $\mathbf{x_8 = (5,1,2.2),}$
    \textbf{compare the train and test RMSE of the two models obtained in (1) and (2). Explain if the results go according to what is expected.}    

    \item \textbf{Consider an MLP to predict the output $\mathbf{y_\text{class}}$ characterized by the weights}
    
    \[
W^{[1]} = \begin{bmatrix}
0.1 & 0.1 \\
0.1 & 0.2 \\
0.2 & 0.1
\end{bmatrix}, \quad
b^{[1]} = \begin{bmatrix}
0.1 \\
0 \\
0.1
\end{bmatrix}, \quad
W^{[2]} = \begin{bmatrix}
1 & 2 & 2 \\
1 & 2 & 1 \\
1 & 1 & 1
\end{bmatrix}, \quad
b^{[2]} = \begin{bmatrix}
1 \\
1 \\
1
\end{bmatrix}
\]

\textbf{the output activation function}

\begin{equation*}
    \text{softmax}(z_c^\text{[out]}) = \frac{e^{z_c^\text{[out]}}}{\sum_{l=1}^{|C|} e^{z_c^\text{[out]}}}
\end{equation*}

\textbf{, no activations on the hidden layer(s) and the cross-entropy loss:}

\begin{equation*}
    \text{CE} = -\sum_{i=1}^{N}\sum_{l=1}^{|C|} t_{l}^{\text{(i)}} \log(Z_{l}^{\text{[out](i)}})
\end{equation*}

\textbf{Consider also that the output layer of the MLP gives the predictions for the classes A, B and C in this order. Perform one
stochastic gradient descent update to all the weights and biases with learning rate $\eta = 0.1$
using the training observation $x_1$.}

\vspace{20pt}
\large{\textbf{Part II}: Programming}\normalsize

\vspace{10pt}
\textbf{Consider the parkinsons.csv dataset (available at the course's webpage), where the goal is
to predict a patient's score on the Unified Parkinson's Disease Rating Scale based on various
biomedical measurements.
To answer question (5), average the performance of the models over 10 separate runs. In each
run, use a different $\mathbf{80-20}$ train test split by setting a \texttt{random\_state = i}, with $\mathbf{i=1...10}$.}


\item \textbf{Train a Linear Regression model, an MLP Regressor with 2 hidden layers of 10
neurons each and no activation functions, and another MLP Regressor with 2 hidden
layers of 10 neurons each using ReLU activation functions. (Use \texttt{random\_state=0} on the
MLPs, regardless of the run). Plot a boxplot of the test MAE of each model.}

\item \textbf{Compare a Linear Regression with a MLP with no activations, and explain the impact
and the importance of using activation functions in a MLP. Support your reasoning with the
results from the boxplots.}

\item \textbf{Using a $\mathbf{80-20}$ train-test split with \texttt{random\_state=0}, use a Grid Search to tune the
hyperparameters of an MLP regressor with two hidden layers (size 10 each). The
parameters to search over are: (i) L2 penalty, with the values $\mathbf{\{0.0001, 0.001, 0.01\}}$; (ii)
learning rate, with the values $\mathbf{\{0.001, 0.01, 0.1\}}$; and (iii) batch size, with the values
$\mathbf{\{32, 64, 128\}}$. Plot the test MAE for each combination of hyperparameters, report the
best combination, and discuss the trade-offs between the combinations.}
\end{enumerate}

\end{document}
