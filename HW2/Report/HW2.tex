\documentclass[12pt]{article}
\usepackage[paper=letterpaper,margin=2cm]{geometry}
\usepackage{amsmath,amssymb,amsfonts}
\usepackage{enumitem}
\usepackage{titling}
\usepackage{multirow}
\usepackage{xcolor}
\usepackage{float}
\usepackage{graphicx}
\usepackage{xcolor}
\definecolor{ISTBlue}{RGB}{0, 139, 255}
\usepackage[colorlinks=true, linkcolor=red]{hyperref}
\usepackage{subcaption} % For subfigures
\usepackage{adjustbox}  % For centering the bottom image
\usepackage{listings}
\usepackage{xcolor} % For setting colors
\usepackage{booktabs} % For better tables
\usepackage{threeparttable} % For table notes

\usepackage{listings}
\usepackage{xcolor}

\definecolor{codegreen}{rgb}{0.0, 0.514, 0.325}      
\definecolor{codegray}{rgb}{0.75, 0.75, 0.75}    
\definecolor{codeblue}{rgb}{0.122, 0.467, 0.706}  
\definecolor{extraLightGray}{rgb}{0.98, 0.98, 0.98}
\definecolor{codepink}{rgb}{0.894, 0.0, 0.443}

\lstdefinestyle{mystyle}{
    backgroundcolor=\color{extraLightGray},
    commentstyle=\color{codegreen},
    keywordstyle=\color{codeblue},
    numberstyle=\tiny\color{codegray},
    stringstyle=\color{codepink},
    basicstyle=\ttfamily\footnotesize,
    breakatwhitespace=false,
    breaklines=true,
    captionpos=b,
    keepspaces=true,
    numbers=left,
    numbersep=5pt,
    showspaces=false,
    showstringspaces=false,
    showtabs=false,
    tabsize=2
}
\lstset{style=mystyle}

\setlength{\droptitle}{-6em}

\begin{document}

\begin{center}
Aprendizagem 2023\\
Homework I --- Group 003\\
(ist1107028, ist1107137)\vskip 1cm
\end{center}

\large{\textbf{Part I}: Pen and paper}\normalsize

\vspace{20pt}
\hspace{-20pt}We collected four positive (P) observations, 
\[
x_1 = (A,0), \quad x_2 = (B,1), \quad x_3 = (A,1), \quad x_4 = (A,0)
\]
and four negative (N) observations,
\[
x_5 = (B,0), \quad x_6 = (B,0), \quad x_7 = (A,1), \quad x_8 = (B,1)
\]


\hspace{-20pt}Consider the problem of classifying observations as positive or negative.

\vspace{10pt}
\begin{enumerate}[leftmargin=\labelsep]
    \item \textbf{Compute the F1-measure of a $k$NN with $k =  5$ and Hamming distance using a
    leave-one-out evaluation schema. Show all calculus.}

    \vspace{10pt}
    We start by calculating Hamming distance between observations. The Hamming distance is the number of positions at which the corresponding symbols are different.\\
    Since we are workin with $k = 5$, we will consider the 5 nearest neighbors of each observation (written in blue).
    \vspace{10pt}
    
    \begin{table}[H]
        \begin{center}
            \begin{tabular}{c|cccccccc}
            & $x_1$ & $x_2$ & $x_3$ & $x_4$ & $x_5$ & $x_6$ & $x_7$ & $x_8$\\ 
            \hline
                $x_1$ & \-- & 2 & \textbf{\textcolor{codeblue}{1}} & \textbf{\textcolor{codeblue}{0}} & \textbf{\textcolor{codeblue}{1}} & \textbf{\textcolor{codeblue}{1}} & \textbf{\textcolor{codeblue}{1}} & 2 \\ 
                $x_2$ & 2 & \-- & \textbf{\textcolor{codeblue}{1}} & 2 & \textbf{\textcolor{codeblue}{1}} & \textbf{\textcolor{codeblue}{1}} & \textbf{\textcolor{codeblue}{1}} & \textbf{\textcolor{codeblue}{0}} \\ 
                $x_3$ & \textbf{\textcolor{codeblue}{1}} & \textbf{\textcolor{codeblue}{1}} & \-- & \textbf{\textcolor{codeblue}{1}} & 2 & 2 & \textbf{\textcolor{codeblue}{0}} & \textbf{\textcolor{codeblue}{1}} \\ 
                $x_4$ & \textbf{\textcolor{codeblue}{0}} & 2 & \textbf{\textcolor{codeblue}{1}} & \-- & \textbf{\textcolor{codeblue}{1}} & \textbf{\textcolor{codeblue}{1}} & \textbf{\textcolor{codeblue}{1}} & 2 \\ 
                $x_5$ & \textbf{\textcolor{codeblue}{1}} & \textbf{\textcolor{codeblue}{1}} & 2 & \textbf{\textcolor{codeblue}{1}} & \-- & \textbf{\textcolor{codeblue}{0}} & 2 & \textbf{\textcolor{codeblue}{1}} \\ 
                $x_6$ & \textbf{\textcolor{codeblue}{1}} & \textbf{\textcolor{codeblue}{1}} & 2 & \textbf{\textcolor{codeblue}{1}} & \textbf{\textcolor{codeblue}{0}} & \-- & 2 & \textbf{\textcolor{codeblue}{1}} \\ 
                $x_7$ & \textbf{\textcolor{codeblue}{1}} & \textbf{\textcolor{codeblue}{1}} & \textbf{\textcolor{codeblue}{0}} & \textbf{\textcolor{codeblue}{1}} & 2 & 2 & \-- & \textbf{\textcolor{codeblue}{1}} \\ 
                $x_8$ & 2 & \textbf{\textcolor{codeblue}{0}} & \textbf{\textcolor{codeblue}{1}} & 2 & \textbf{\textcolor{codeblue}{1}} & \textbf{\textcolor{codeblue}{1}} & \textbf{\textcolor{codeblue}{1}} & \-- \\ 
            \end{tabular}
            \caption{Hamming distance between observations}
        \end{center}
    \end{table}

    Now that we have the Hamming distance between all observations, we must identify if the prediction is correct or not. We will consider the majority class of the 5 nearest neighbors for each observation.
    
    \vspace{10pt}
    \underline{Example}: For $x_1$, the 5 nearest neighbors are $x_3$ and $x_4$ (which are positive), $x_5$, $x_6$ and $x_7$ (which are negative). The majority class is negative, therefore the prediction is incorrect.
    
    \newpage
    We apply the same logic for the rest of the classes, ending up with the following table:

    \begin{table}[H]
        \begin{center}
            \begin{threeparttable}
            \begin{tabular}{c|c|c|c}
                Observation & True Value & Prediction & Confusion Matrix Terminology\\
                \hline
                $x_1$ & P & N & FP\\
                $x_2$ & P & N & FP\\
                $x_3$ & P & P & TP\\
                $x_4$ & P & N & FN\\
                $x_5$ & N & P & FP\\
                $x_6$ & N & P & FP\\
                $x_7$ & N & P & FP\\
                $x_8$ & N & N & TN\\
            \end{tabular}
            \begin{tablenotes}
                \small
                \item[]
                \item[P - Positive observation; N - Negative observation]  
                \item[TP - True Positive; TN - True Negative; FP - False Positive; FN - False Negative] 
                \item[] 
            \end{tablenotes}
        \end{threeparttable}
            \caption{Predictions for each observation}
        \end{center}
    \end{table}

    With this table, we can know calculate the Precision, Recall and F1-measure using the following formulas:

    \begin{equation}\label{precision}
        \text{Precision} = \frac{\text{True Positives}}{{\text{True Positives} +
        \text{False Positives}}}
    \end{equation}

    \begin{equation}\label{recall}
        \text{Recall} = \frac{\text{True Positives}}{{\text{True Positives} + \text{False Negatives}}}
    \end{equation}

    \begin{equation}\label{f1}
        \text{F1-measure} = 2 \times \frac{{\text{Precision} \times \text{Recall}}}{{\text{Precision} + \text{Recall}}}
    \end{equation}

    \vspace{10pt}
    Replacing the corresponding values in the formulas, we get:

    \vspace{10pt}
    for Precision \eqref{precision} and Recall \eqref{recall}:

    \begin{equation*}
        \text{Precision} = \frac{1}{1 + 5} \approx 0.1667 \quad \quad
        \text{Recall} = \frac{1}{1 + 1} = 0.5
    \end{equation*}

    F1-measure \eqref{f1}:

    \begin{equation}
        \text{F1-measure} = 2 \times \frac{0.1667 \times 0.5}{0.1667 + 0.5} \approx 0.25
    \end{equation}

    \vspace{10pt}
    \item \textbf{Propose a new metric (distance) that improves the latter's performance (i.e., the
    F1-measure) by three fold.} 

\end{enumerate}

\end{document}
